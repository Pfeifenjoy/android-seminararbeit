%!TEX root = ../dokumentation.tex

\chapter{Einleitung}


Im Rahmen der Vorlesung \enquote{Entwicklung mobiler Applikationen - Thema Android} an der DHBW Mannheim, ist es vorgesehen ein Projekt durchzuführen, mit der Zielsetzung eine lauffähige App für das Betriebssystem Android zu erstellen.
\\Es ist den Studenten freigestellt, welche Android-Versionen die erstellte App unterstützen soll.\enquote{Piccer} ist ab der Android-Version 4.4 (API Level 19) lauffähig bis Android 5.1 (API Level 22).
\\Als Testgeräte wurden ein \enquote{Motorola Moto X} der zweiten Generation mit dem Betriebssystem Android 5.1 (Lollipop), sowie ein \enquote{Samsung Galaxy S3} mit dem Betriebssystem CyanogenMod 11.0, welches Android 4.4 entspricht, genutzt.


\section{Aufgabenstellung}
\begin{enumerate}
\item Erstellen Sie eine App, die Fotos über ein Menü aufnimmt und aus der Fotogalerie lädt.
\item Speichern Sie das aufgenommene Foto in einem speziellen Verzeichnis mit aktuellem Datum und Uhrzeit.
\item Stellen Sie die aufgenommenen Fotos in einer Liste dar. In der Liste soll der Name des Fotos und das erstellte Datum angezeigt werden.
\item Wenn man auf ein Foto tippt, dann wird eine zweite Activity gestartet, indem das gemachte Foto in einer ImageView angezeigt wird.
\item Eine weitere Option soll das Versenden des Fotos per E-Mail ermöglichen.
\item Eine weitere Option soll das Speichern des Fotos in die Fotogalerie ermöglichen.
\end{enumerate}

\newpage

\section{Funktionen von \enquote{Piccer}}
\textbf{\enquote{Piccer} kann:}
\begin{enumerate}
\item Fotos aufnehmen und aus der Galerie laden.
\item Thumbnails in einer Liste mit Datum, Uhrzeit und Titel anzeigen. 
\item Mehrere Fotos, die in der Liste ausgewählt werden, löschen, in der Galerie speichern oder teilen.
\item Die Anordnung der Liste kann verändert werden. Entweder ist das erste Foto, das angezeigt wird, das zuletzt hinzugefügte oder das zu erst hinzugefügte.
\item Ein ausgewähltes Foto in der ImageView als zweite Activity anzeigen.
\item Im ImageView kann das ausgewählte Foto versendet, gelöscht oder in der Galerie gespeichert werden.
\item Außerdem kann der Titel geändert werden.

\end{enumerate}


